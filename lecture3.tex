\documentclass{ximera}



% MATH -----------------------------------------------------------
\newcommand{\nats}{\mathbb N}
\newcommand{\ints}{\mathbb Z}
\newcommand{\rats}{\mathbb Q}
\newcommand{\reals}{\mathbb R}
\newcommand{\complex}{\mathbb C}
\newcommand{\powerset}{\mathscr P}

%-------------------------------------------------------------


\begin{document}
\begin{abstract}
 \end{abstract}

    \title{Separable Equations}
    
    \maketitle


 A 1st order differential equation is \underline{separable} if it can be written as $h(y)y'=g(x)$.\\

  From this form, integrate both sides with respect to the independent variable  (and use $dy=y\,'\,dx$)  to solve for the general solution:\\

 $\displaystyle h(y)y'=g(x)\rightarrow$\\

 $\displaystyle \int h(y)\,y\,' \,dx=\int g(x)\,dx\rightarrow$ \\

 $\displaystyle \int h(y)\,dy=\int g(x)\,dx\rightarrow $\\

 $H(y)=G(x)+C$ where $H(y)$ is an antiderivative of $h(y)$ and $G(x)$ is an antiderivative of $g(x)$.\\

 $H(y)=G(x)+C$ is called an $``$implicit" general solution; if you can solve for $y$, the result is called an $``$explicit" general solution.\\   \\


 Example:  Let's solve the initial value problem (IVP) $\displaystyle y\,'=\frac{2\cos{(2t)}}{y^2}$, $y(0)=1$.\\

 First we $``$separate" the variables:  $y^2 \,y\,'=2\cos{(2t)}$.\\

 Then we integrate both sides with respect to $t$:  $\displaystyle \frac{1}{3}y^3=\sin{(2t)}+C$.

 Then we solve for $y$:  $y=\sqrt[3]{3\sin{(2t)}+C}$ (Note: $``3C$" can be relabeled as $``C$")\\

 This is an explicit general solution.  To find the solution that matches the IVP we use the initial condition $y(0)=1$:   $1=\sqrt[3]{0+C}$, so $C=1$.\\

 The IVP solution is $y=\sqrt[3]{3\sin{(2t)}+1}$.

\newpage

 Another example:  Let's solve the equation $\displaystyle y\,'=\frac{y+y^3}{x}$.\\

 First observe that $y=0$ is a solution.  Let's look for non-trivial solutions.\\

 Separate the variables:   $\displaystyle \frac{1}{y+y^3}y\,'=\frac{1}{x}$.\\

 Integrate:  $\displaystyle \int \frac{1}{y(1+y^2)}\,y\,' dx=\int \frac{1}{x}\,dx$.\\

 Using partial fractions,  $\displaystyle \int \frac{1}{y}-\frac{y}{1+y^2} \, dy=\ln{|x|}$.\\

 So $\displaystyle \ln{|y|}-\frac{1}{2}\ln{(1+y^2)}=\ln{|x|}+C$ (OR $y=0$).\\

 That's an implicit solution.  By log rules, we can rewrite it as $\displaystyle \ln{\left(\frac{|y|}{\sqrt{1+y^2}}\right)}=\ln{|x|}+C$.\\

 Exponentiating both sides,  $\displaystyle \frac{|y|}{\sqrt{1+y^2}}= e^{\ln{|x|}+C}=e^C |x|$ which can be expressed as \\


 $\displaystyle \frac{y}{\sqrt{1+y^2}}=Cx$  (Note: $``\pm e^C$" can be relabeled as $``C$" and we even allow for $C=0$).\\

 Finally, $\displaystyle y^2 = Cx^2(1+y^2)\rightarrow$  $y^2 - Cx^2 y^2=Cx^2\rightarrow$  $y^2(1-Cx^2)=Cx^2\rightarrow$ $y=\pm \sqrt{\frac{Cx^2}{1-Cx^2}}$ where $C$ is a non-negative constant.  \\

 We can even express this as $y=0$ OR $\displaystyle y=\pm \sqrt{\frac{x^2}{C-x^2}}=\pm \frac{x}{\sqrt{C-x^2}}$  where $C$ is a positive constant.\\

 Note:  Sometimes we cannot get to an explicit solution, and we just leave it implicit.\\ \\

 One can show that the IVP $\displaystyle y\,'=\frac{y+y^3}{x}$, $y(3)=-0.75$ has the solution $\displaystyle y=-\frac{x}{\sqrt{25-x^2}}$.  \\

 But on what intervals would this be a valid solution?  \\ 

 First of all, by looking at the de, $x\neq 0$, so $x>0$ to include $x=3$. Next by looking at the domain of this given solution, we see that $x<5$. \\

 Putting this all together, this IVP solution is valid on any open interval contained in $(0,5)$, including $(0,5)$.





\end{document}