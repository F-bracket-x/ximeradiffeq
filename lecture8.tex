\documentclass{ximera}



% MATH -----------------------------------------------------------
\newcommand{\nats}{\mathbb N}
\newcommand{\ints}{\mathbb Z}
\newcommand{\rats}{\mathbb Q}
\newcommand{\reals}{\mathbb R}
\newcommand{\complex}{\mathbb C}
\newcommand{\powerset}{\mathscr P}

%-------------------------------------------------------------


\begin{document}
\begin{abstract}
 \end{abstract}

    \title{First Order Autonomous}
    
    \maketitle


 A first order differential equation that can be written in the form $\dfrac{dy}{dt}=f(y)$ is \underline{autonomous}.\\

 Constant functions $y=C$ such that $f(C)=0$ satisfy the de on $(-\infty,\infty)$. These constant solutions are called \underline{equilibrium solutions}.\\

  (Note:  Same definition applies for $y'=f(t,y)$.  Any constant function $y=C$ such that $f(t,C)=0$ is called an equilibrium solution.)\\


  Here are a few useful properties of first order autonomous differential equations

 \begin{enumerate}

\item They are separable.
\item The slopes in the direction field only depend on $y$.
\item The general solution is invariant under horizontal translations.\\

 \end{enumerate}


 Example:  Let's consider resource-limited population growth.  Let $p(t)$ represent the size of a population at time $t$.  When resources are restricted we can employ a $``$competition factor" $-bp^{2}$, so that $\displaystyle \frac{dp}{dt}=rp-bp^{2}=rp\left(1-\dfrac{p}{K}\right)$, where $K=\dfrac{r}{b}$.\\

 This model is called a \underline{logistic equation} and is a nice example of an autonomous equation.  This differential equation has equilibrium solutions at $p=0$ and $p=K$.  We call $K$ the $``$carrying capacity" of the model and $r$ the $``$intrinsic growth rate."

\newpage

 Let's determine the $``$stability" of these equilibrium values.  Assume $r>0$ and $K>0$ (negative values would make no sense for this model).\\


  To do this we plot $\dfrac{dp}{dt}$ versus $p$ for $\dfrac{dp}{dt}=rp\left(1-\dfrac{p}{K}\right)$\\



\begin{figure}[h]
    \centering

\includegraphics[scale=0.6]{pprimeversusp.pdf}

    \caption{Phase plane}
\end{figure}


 The graph is a parabola with $x$-intercepts $(0,0)$ and $(K,0)$.\\

 Note: If $0<p<K$ then $\dfrac{dp}{dt}>0$ and $p$ is increasing.  If $p>K$ then $\dfrac{dp}{dt}<0$ and $p$ is decreasing.  So if $p$ is close to $K$ then $p$ is $``$pushed" towards $K$.  A similar analysis shows that if $p$ is close to $0$ it will be $``$pushed" away from $0$. \\ \\  (While we get $\dfrac{dp}{dt}<0$ when $p<0$, it's not relevant to the model as $p$ cannot be negative.)\\

 We can use arrows to indicate how $p$ is $``$pushed."  Here it is with the logistic equation:\\

\begin{figure}[h]
    \centering

 \includegraphics[scale=0.6]{pprimeversuspwitharrows.pdf}

    \caption{Phase plane with arrows}
\end{figure}


 For $y'=f(y)$, an equilibrium solution $y=C$ is called an \underline{asymptotically stable equilibrium solution} if small perturbations (changes) are $``$pushed" back towards $y=C$.  An equilibrium solution $y=C$ is called an \underline{asymptotically unstable equilibrium solution}  if small perturbations will $``$push" $y$ away from $y=C$.  An equilibrium solution $y=C$ where the asymptotic behavior depends on which direction the perturbation is called an\\ \underline{asymptotically semistable equilibrium solution}.\\

 With the arrows, just the horizontal axis is called the \underline{phase line}.  Often these are drawn vertically.

\newpage

 Example: Find and classify the equilibrium solutions of $\dfrac{dy}{dt}=(1-y)(2-y)$.

\begin{figure}[h]
    \centering

 \includegraphics[scale=0.6]{anotherwitharrows.pdf}

    \caption{Another phase plane with arrows}
\end{figure}

 The equilibrium solutions are $y=1$ and $y=2$.  Since $y'$ is positive outside of the interval $[1,2]$ and negative inside $(1,2)$, $y=1$ is a stable solution and $y=2$ is an unstable solution.\\

 Another example: Find and classify the equilibrium solutions of $\dfrac{dy}{dt}=y(2-y)^{2}$.

\begin{figure}[h]
    \centering

\includegraphics[scale=0.6]{yetanotherwitharrows.pdf}

    \caption{Yet another phase plane with arrows}
\end{figure}

The equilibrium solutions are $y=0$ and $y=2$.  $y=0$ is an unstable solution  and $y=2$ is a semistable solution since it is stable from one side (in this case from below) while unstable from the other side (in this case from above).

\newpage


 Yet another example:  Find and classify the equilibrium solutions of $\dfrac{dy}{dt}=\sin{(y)}$.

\begin{figure}[h]
    \centering


 \includegraphics[scale=1]{annoyinggraphic.pdf}

    \caption{One more phase plane with arrows}
\end{figure}

 Unstable equilibria exist at $t=0, \pm 2\pi,\pm 4\pi,...$

 Stable equilibria exist at $t=\pm \pi,\pm 3\pi,...$


\end{document}
