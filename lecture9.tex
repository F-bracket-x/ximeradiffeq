\documentclass{ximera}



% MATH -----------------------------------------------------------
\newcommand{\nats}{\mathbb N}
\newcommand{\ints}{\mathbb Z}
\newcommand{\rats}{\mathbb Q}
\newcommand{\reals}{\mathbb R}
\newcommand{\complex}{\mathbb C}
\newcommand{\powerset}{\mathscr P}

%-------------------------------------------------------------


\begin{document}
\begin{abstract}
 \end{abstract}

    \title{First Order Applications, Part I}
    
    \maketitle

     There are many applications for first order differential equations. \\


 The classic \emph{exponential model} $y'(t)=ky$, $y(t_{0})=y_{0}$ can be used to model unrestricted population growth $(k>0)$, radioactive decay $(k<0)$, etc.\\

 We have already seen the \emph{logistic model} $y'(t)=ry\left(1-\frac{y}{K}\right)$, $y(t_{0})=y_{0}$ which can model resource-restricted population growth, the spreading of news or a rumor, etc.\\

 Here is a twist on the exponential model:  \\

 Suppose that a radioactive substance has decay constant $k>0$ (reported as a positive number, so use $y'=-ky$ in the model).  At the same time the substance is being produced at a constant rate of $a$ units of mass per unit time. Let $y(t)$ be the amount of the substance at time $t$ and $y(0)=y_{0}$ be the initial mass. \\

 First we derive a differential equation:  $y'(t)=\mbox{ the rate of increase in }y\mbox{ }-\mbox{ the rate of decrease in }y.$\\

 So $y'(t)=a-ky$.  Either $\displaystyle y=\frac{a}{k}$ (equilibrium solution) or  $\displaystyle \frac{1}{a-ky}y'=1\rightarrow$ \\  $\displaystyle -\frac{1}{k}\ln{|a-ky|}=t+C\rightarrow$  $\displaystyle a-ky=\pm e^{-kt+C}\rightarrow$  $\displaystyle a-ky=Ce^{-kt}\rightarrow$  $\displaystyle y=\frac{a}{k}+Ce^{-kt}$ \\(which includes the constant equilibrium solution if we allow $C=0$).\\

 In terms of $y(0)=y_0$ we get $\displaystyle y=\frac{a}{k}+\left(y_0-\frac{a}{k}\right)e^{-kt}$.  In particular, if $y_{0}>a/k$ then $y$ decreases toward the equilibrium solution, and if $y_{0}<a/k$ then $y$ increases towards the equilibrium solution.

 \newpage

 Another common model is \emph{Newton's Law of Cooling}:  If an object of temperature $T(t)$ at time $t$ is in a medium of temperature $T_{m}(t)$ at time $t$  then the rate of change in $T(t)$ is proportional to the $\Delta T = T(t)-T_{m}(t)$.  When $T(t)>T_{m}(t)$ we have $T'(t)<0$ (cooling)  and when $T(t)<T_{m}(t)$ we have $T'(t)>0$ (warming).\\


 So $T(t)$ satisfies $T'=-k(T-T_{m})$ where $k>0$.\\

 Note: If $T_m (t)$ is a constant function this model is essentially a shifted exponential decay model.\\ \\


\emph{Mixing applications}:  Involve the amount $y$ of something that is mixed,  and \\ \\ $y'=$ the rate of $y$ coming in MINUS the rate of $y$ going out, $y(0)=$ the initial amount.\\


 Example:   A $500$ L tank initially contains $2$ kg of salt dissolved in $100$ L of water.  Saltwater that contains $0.1$ kg of salt per liter is pumped into the tank at $4$ liters per minute and at the same time water is drained from the tank at $2$ liters per minute.  Assume the tank is always kept uniformly mixed. Calculate the amount of salt in the tank when the tank begins to overflow.\\

 Let $t$ represent time in minutes (where $t=0$ corresponds with the initial state of the tank). First note that after $t$ minutes the volume in the tank is $V(t)=100+2t$ L  so the tank begins to overflow at $t=200$ (each minute a net gain of 2 L of saltwater goes into the tank and the tank has $400$ L of space for the incoming saltwater). \\

 Let $y(t)$ be the amount of salt in kg at time $0\leq t\leq 200$.  At time $t=0$, $y(0)=2$.\\

 $y'(t)=$ rate of salt in $\;-\;$ rate of salt out  $\displaystyle =\left(\frac{0.1 \;\mathrm{kg}}{\mathrm{L}}\right)\left(\frac{4\; \mathrm{L}}{\mathrm{min} }\right)-\left(\frac{y(t)\; \mathrm{kg}}{100+2t\;\mathrm{L}}\right)\left(\frac{2\;\mathrm{L}}{\mathrm{min}}\right)$.\\

 So we have an IVP (simplified):   $\displaystyle y'=\frac{2}{5}-\frac{y}{50+t}$, $y(0)=2$.  While not separable, it is linear:\\

 In standard form, $\displaystyle y'+\frac{1}{t+50}\,y=\frac{2}{5}$.   An integrating factor would be $\mu (t)=e^{\ln{(t+50)}}=t+50$.

 \newpage
 

 So  $\displaystyle (t+50)\, y'+ y=\frac{2}{5}t+20\rightarrow $  $\displaystyle \left[(t+50)y\right]'=\frac{2}{5}t+20\rightarrow $  $\displaystyle (t+50)y=\frac{1}{5}t^2+20t+C\rightarrow $\\

 $\displaystyle y=\frac{\frac{1}{5}t^2+20t+C}{t+50}$.  Using $y(0)=2$ we see that $C=100$. So $\displaystyle y=\frac{\frac{1}{5}t^2+20t+100}{t+50}$ \\


 and at the time of overflow,  $\displaystyle y(200)=\frac{\frac{1}{5}(200)^2+20(200)+100}{200+50}=\frac{12,100}{250}=48.4$ kg of salt is in the tank.\\


\begin{figure}[h]
    \centering

 \includegraphics[scale= 1]{MixingDirectionField.pdf}

    \caption{A direction field with a solution curve}
\end{figure}



\end{document}
